This paper investigates the impact of clock deviation within a component-port-connector architecture model. The OMNeT++ framework defines the behavior of system components and connectors. The resulting model is then translated into the PRISM language, utilizing Probabilistic Decision Tree rules derived from the OMNeT++ simulation trace. Requirements related to synchronized clocks, expressed in Probabilistic Computation Tree Logic (PCTL), are verified against two models constructed in PRISM \cmt{PA}: a reference model and one generated by learning decision trees. A use case is then implemented in both formalisms to demonstrate the effectiveness of the proposed approach for validation. 

The models incorporate parameters inspired by real-world phenomena observed in standard specifications, including product manufacturing variations and operating temperature changes. This parameterized construction allows practitioners to perform customized validation and verification. Our results demonstrate that building stochastic models from simulated models leads to more efficient models with fewer states and transitions compared to PRISM models.

In future research endeavors, we plan to formally capture the bi-simulation between the constructed models and the learned abstract model. This task presents a significant challenge due to the potential differences in the state spaces of the input and abstract models. Additionally, we aim to develop an automated approach for feature selection during the learning process. The current approach, while promising, requires human intervention in choosing the features used for model construction. \cmt{Finally, to enhance realism, we plan to incorporate more precise communication delays. We aim to achieve this by utilizing electronic tools for delay measurement or more advanced simulation techniques.}