\documentclass[3p,times,preprint,number,round]{elsarticle}
%\documentclass[preprint,12pt]{elsarticle}

\newcommand\hmmax{0}
  \newcommand\bmmax{3}
%\usepackage{algorithm2e}
%\usepackage{algorithm}
%\usepackage{algpseudocode}
%\usepackage{algorithmic}
 \usepackage{mathtools}
\usepackage{makeidx}
    \usepackage{float}
    \usepackage{graphicx}
\usepackage{makeidx}  % allows for indexgeneration
\usepackage[english]{babel} % un troisième package
\usepackage{amssymb}
\usepackage{amsmath}
\usepackage{syntax}
\usepackage{multirow}
\usepackage{array}
\usepackage[table]{xcolor}
\usepackage{enumitem}
\usepackage{booktabs}
\usepackage{listings}
\usepackage{verbatim}
%\usepackage{caption}
\usepackage{subcaption}
\usepackage{courier}
%\usepackage{mathrsfs}
%\usepackage[authoryear]{natbib}
%\usepackage{amsthm}
\usepackage{amsthm}
%\usepackage{nath}

\usepackage{lmodern}
%\usepackage{bm}
\usepackage{tikz}
%\usepackage[]{algorithm2e}
%\usepackage{algorithm}


%\usepackage{algorithm}   % For math symbols
%\usepackage[noend]{algpseudocode}
%\usepackage{algorithm2e}
%\usepackage{algorithmic}
\usepackage[ruled,lined,linesnumbered]{algorithm2e}
%\usepackage{algorithm}
%\usepackage{algorithmic}
%\usepackage{algpseudocode}
\usepackage[hyphens]{url}
%\usepackage{breakurl}
%\PassOptionsToPackage{hyphens}{url}
\usepackage{hyperref}
\hypersetup{
  colorlinks,
  citecolor= 	JungleGreen,
  linkcolor= 	JungleGreen,
  urlcolor= 	JungleGreen}
  
  \usepackage{tabularx}
\usepackage{rotating}
\usepackage[T1]{fontenc}
\usepackage[utf8]{inputenc}
%\usepackage{amsmath,amsfonts,amssymb}
%\usepackage{mathastext}
\usepackage[bitstream-charter]{mathdesign}
%\usepackage[T1]{fontenc}
\usepackage{booktabs}
\normalfont
\usepackage{listings}     
\usepackage{lstautogobble}  % Fix relative indenting
\usepackage{color}          % Code coloring
\usepackage{zi4}            % Nice font
\definecolor{bluekeywords}{rgb}{0.13, 0.13, 1}
\definecolor{greencomments}{rgb}{0, 0.5, 0}
\definecolor{redstrings}{rgb}{0.9, 0, 0}
\definecolor{graynumbers}{rgb}{0.5, 0.5, 0.5}
\definecolor{forestgreen}{rgb}{0.0, 0.5, 0.0}

\definecolor{commentgreen}{RGB}{2,112,10}
\definecolor{eminence}{RGB}{108,48,130}
\definecolor{weborange}{RGB}{255,165,0}
\definecolor{frenchplum}{RGB}{129,20,83}

\usepackage[hyphens]{url}
\usepackage{hyperref}
\hypersetup{
  colorlinks,
  citecolor= 	blue,
  linkcolor= 	blue,
  urlcolor= 	blue}
    \usepackage{breakurl}
    
%% set the starting page if not 1
%\firstpage{1}
\DeclareCaptionFont{white}{\color{white}}
\DeclareCaptionFormat{listing}{%
    \colorbox{black}{\parbox{\dimexpr\textwidth-2\fboxsep}{\textbf{\textcolor{white}{#1#2#3}}}}}
\captionsetup[lstlisting]{format=listing,labelfont=white,textfont=white}
%% Give the name of the journal

\usepackage{lineno}

\usepackage{tikz}
\usetikzlibrary{matrix,arrows.meta,arrows,automata,backgrounds}
\tikzset{%
  >={Latex[width=2mm,length=2mm]},
  % Specifications for style of nodes:
            base/.style = {rectangle, rounded corners, draw=black,
                           minimum width=2cm, minimum height=1cm,
                           text centered, font=\sffamily},
  activityStarts/.style = {base, fill=blue!30},
       startstop/.style = {base, fill=red!30},
    activityRuns/.style = {base, fill=green!30},
         process/.style = {base, minimum width=2.5cm, fill=orange!15,
                           font=\ttfamily},
}

\newcommand*\circled[1]{\tikz[baseline=(char.base)]{
    \node[shape=circle, draw, inner sep=1pt, 
        minimum height=12pt] (char) {#1};}}
        \ttfamily

\newcommand{\sensinact}{\texttt{sensi\textbf{N}act}}

\newcommand{\ecircle}[1] {\textcircled{{\small #1}}}


\newcommand{\fig}[1]{Figure \ref{#1}}

\newcommand{\commenting}[1] {\textcolor{red}{#1}}

\newcommand{\quot}[1] {``#1''}

\newcommand{\emath}[1] { $#1$ }

\newcommand{\emathtt}[1] { $\mathtt{#1}$ }

\newcommand{\msym}[1] { $\mathscr{#1}$ }

\newcommand{\alg}[1] {Algorithm \ref{#1}}

\newcommand{\lst}[1] {Listing \ref{#1}}

\newcommand{\ie}{i.e., }
\newcommand{\eg}{e.g., }
\newcommand{\thisrelation}{${\rm I\!R}$}

\newtheorem{theorem}{Theorem}%[section]
\newtheorem{lemma}[theorem]{Lemma}
\newtheorem{proposition}[theorem]{Proposition}
\theoremstyle{definition}
\newtheorem{mydef}{Definition}
\newtheorem{example}{Example}
%\volume{00}
\newcommand{\gparrow}[1] {\lhook\joinrel\xrightarrow{#1} }
%\firstpage{1}
\newcommand{\acisiot} {s\textbf{A}fety and se\textbf{C}ur\textbf{I}ty as\textbf{S}urrance for critical \textbf{IoT} systems}
\newcommand{\G}[1]{\textbf{#1}}

\newcommand{\hermesdesign} {\G{H}uman-C\G{E}ntric Collabo\G{R}ative Architectural Decision-\G{M}aking for S\G{E}cure \G{S}ystem Design}

\journal{Elsevier}

%\newcommand*\circled[1]{\tikz[baseline=(char.base)]{\node[shape=circle,draw,inner sep=2pt] (char) {#1};}}
\usepackage[many]{tcolorbox}  
\usepackage{setspace}               % for LINE SPACING
\usepackage{multicol}               % for MULTICOLUMNS

  
\usetikzlibrary{calc,shadows.blur}
\def\llbracket{[\![}
\def\rrbracket{]\!]}

\newcommand{\sset}[1] {\llbracket #1 \rrbracket}
\newtcolorbox{boxD}{
    colback = gray!5!white, 
    colframe = black, 
    boxrule = 0pt, 
    toprule = 3pt, % top rule weight
    bottomrule = 3pt % bottom rule weight
}

\newtcolorbox{boxC}{
    colback = blue!0!white,  % background color
    boxrule = 0pt  % no borders
}



%\runauth{}
%\jid{procs}
\makeindex

%\jnltitlelogo{\small Information Sciences}
\normalsize

\usepackage{amssymb}

\setcitestyle{square}

\SetSymbolFont{operators}   {normal}{OT1}{cmr} {m}{n}
\SetSymbolFont{letters}     {normal}{OML}{cmm} {m}{it}
\SetSymbolFont{symbols}     {normal}{OMS}{cmsy}{m}{n}
\SetSymbolFont{largesymbols}{normal}{OMX}{cmex}{m}{n}
\SetSymbolFont{operators}   {bold}  {OT1}{cmr} {bx}{n}
\SetSymbolFont{letters}     {bold}  {OML}{cmm} {b}{it}
\SetSymbolFont{symbols}     {bold}  {OMS}{cmsy}{b}{n}
\SetSymbolFont{largesymbols}{bold}  {OMX}{cmex}{m}{n}

\SetMathAlphabet{\mathbf}{normal}{OT1}{cmr}{bx}{n}
\SetMathAlphabet{\mathsf}{normal}{OT1}{cmss}{m}{n}
\SetMathAlphabet{\mathit}{normal}{OT1}{cmr}{m}{it}
\SetMathAlphabet{\mathtt}{normal}{OT1}{cmtt}{m}{n}
\SetMathAlphabet{\mathbf}{bold}  {OT1}{cmr}{bx}{n}
\SetMathAlphabet{\mathsf}{bold}  {OT1}{cmss}{bx}{n}
\SetMathAlphabet{\mathit}{bold}  {OT1}{cmr}{bx}{it}
\SetMathAlphabet{\mathtt}{bold}  {OT1}{cmtt}{m}{n}

\usepackage{enumitem}
\linenumbers

\newcommand{\eclipse}[1] {\textcolor{eminence}{\texttt{\textbf{#1}}}}


\newcommand{\cmt}[1] {\textcolor{blue}{#1}}


\newcommand{\gtext}[1] {\textcolor{forestgreen}{#1}}

\newcommand{\keyw}[1] {\texttt{\textbf{#1}}}

\newcommand{\includesection}[1] {\label{#1}\begin{sloppypar}\input{#1}\end{sloppypar}}


\newcommand\setItemnumber[1]{\setcounter{enumi}{\numexpr#1-1\relax}}
\usepackage{framed}

%%%%%%%%%%%%%%%%%%%%%%%%%%%%%%%%%%%%%%
\usetikzlibrary{calc,shadows.blur}
\tcbuselibrary{skins}
\newtcolorbox{resp}[2][]{%
  enhanced jigsaw,
  colback=gray!20!white,%
  colframe=gray!80!black,
  size=small,
  boxrule=1pt,
  title=#2,
  halign title=flush center,
  coltitle=black,
  drop shadow=black!50!white,
  attach boxed title to top left={xshift=1cm,yshift=-\tcboxedtitleheight/2,yshifttext=-\tcboxedtitleheight/2},
  minipage boxed title=3cm,
  boxed title style={%
    colback=white,
    size=fbox,
    boxrule=1pt,
    boxsep=2pt,
    underlay={%
      \coordinate (dotA) at ($(interior.west) + (-0.5pt,0)$);
      \coordinate (dotB) at ($(interior.east) + (0.5pt,0)$);
      \begin{scope}
        \clip (interior.north west) rectangle ([xshift=3ex]interior.east);
        \filldraw [white, blur shadow={shadow opacity=60, shadow yshift=-.75ex}, rounded corners=2pt] (interior.north west) rectangle (interior.south east);
      \end{scope}
      \begin{scope}[gray!80!black]
        \fill (dotA) circle (2pt);
        \fill (dotB) circle (2pt);
      \end{scope}
    },
  },
  #1,
}
%\usepackage{calrsfs}
%\DeclareMathAlphabet{\pazocal}{OMS}{zplm}{m}{n}
%\newcommand{\Lb}[1]{\pazocal{#1}}
\newtcolorbox{boxF}{
    colback = yellow!5!white,
    enhanced,
    boxrule = 1.5pt, 
    colframe = white, % making the base for dash line
    borderline = {1.1pt}{0pt}{main, dashed} % add "dashed" for dashed line
}


\usepackage[font=large,labelfont=bf]{caption} % Load caption package


\newtcolorbox{graybox}{colback=gray!20,  top=0pt, bottom=0pt, left=2mm, right=2mm}


\begin{document}

\begin{frontmatter}

%\dochead{}
%Formalizing the Detection and Mitigation of Clock Deviation in Challenging Physical and Environmental Conditions for Synchronous and Asynchronous Communication Styles

%Detection and Mitigation of Clock Deviation in Challenging Physical and Environmental Conditions: A Toolbox Implementation Utilizing PRISM, OMNeT++, and Probabilistic Decision Trees

%\title{Verification and Simulation: Detection and Mitigation of Clock Deviation}


\title{\cmt{Detection and Mitigation of Clock Deviation in the Verification \& Validation of Drone-aided Lifting Operations}}


%\title{On the Applicability of Model Checking and Simulation for Detection and Mitigation of Clock Deviation}
%in Challenging Physical and Environmental Conditions}

\author[aut2]{Abdelhakim Baouya\corref{cor1}}
\ead{abdelhakim.baouya@irit.fr}

\author[aut2]{Brahim Hamid}
\ead{brahim.hamid@irit.fr}

\author[aut3]{Otmane Ait Mohamed}
\ead{otmane.aitmohamed@concordia.ca}

\author[aut4]{Saddek Bensalem}
\ead{saddek.bensalem@univ-grenoble-alpes.fr}

\cortext[cor1]{Corresponding author at : IRIT, Université de Toulouse, CNRS, UT2, 118 Route de Narbonne, 31062 Toulouse Cedex 9, France}


\address[aut2]{IRIT, Université de Toulouse, CNRS, UT2, 118 Route de Narbonne, 31062 Toulouse Cedex 9, France}

\address[aut3]{Concordia University, Montréal, Canada}


\address[aut4]{Université Grenoble Alpes, VERIMAG, CNRS, Grenoble, France}
% using a component-port-connector architecture model
\begin{abstract}
Modern cyber-physical systems, relying on diverse computation logics, communication protocols, and technologies, are susceptible to environmental phenomena \cmt{and production errors} that can significantly impact system behavior. \cmt{Ensuring resilience in these systems necessitates considering these factors during the high-level design stages to enable accurate functional forecasting}. \cmt{This paper presents an approach that models clock deviation's effects within physical and environmental conditions to perform verification \& validation}. We employ the OMNeT++ simulation framework to define the behavior of system \cmt{in components-port-connectors fashion}. The approach leverages Probabilistic Decision Tree rules, derived from the OMNeT++ simulation chart. The resulting rule-based model is then interpreted in the PRISM language for automated model verification. \cmt{To validate our approach, we investigate how clock deviations influence the correctness of drone-aided lifting operations, serving as a representative application scenario}. The research examines clock deviations from multiple sources, including standard specifications, product manufacturing variations, and operating temperature changes. Our examination explores the potential of validation through simulation and model checking, while also studying the approach's effectiveness in terms of scalability.
\end{abstract}
%% expressed using probabilistic computation tree logic (PCTL) to ensure the reliability of data exchange. 
\begin{keyword}
 System architecture\sep Clock drifts\sep Decision Trees \sep Formal methods \sep Drones.
\end{keyword}
\end{frontmatter}

\tableofcontents

\newpage
\section{Introduction}
\includesection{introduction}


\section{\cmt{Literature review}}
\includesection{relatedwork}


\section{Background}
\includesection{background}


\section{Approach}
\includesection{approach}








% \section{Formalization}
% \label{formalization}
% This section presents the formalization of Probabilistic Decision Trees (PDTs). PDTs are crucial for constructing state-based models by leveraging rule sets. 




% \subsection{OMNeT++ }
% \includesection{omnet}



\section{Communication in Challenging Physical and Environmental Conditions}
\label{communication}
\includesection{communication}

 
\section{Experiments}
\includesection{evaluation}


\section{Discussion}
\includesection{discussion}






\section{Conclusion}
\includesection{conclusion}

% \subsection*{Acknowledgement}
% The research leading to the presented results was conducted within the research profile of \acisiot ~(\textsc{Acis-IoT}) supported by the Centre National de la Recherche Scientifique (CNRS) and \hermesdesign~(\textsc{Hermes-Design}) supported by Institut de Cybersécurité d'Occitanie (ICO). 

%\bibliographystyle{model1a-num-names}
%\bibliographystyle{plainnat}
%\bibliographystyle{model3-num-names}

\bibliographystyle{elsarticle-num-names}
\bibliography{references}


\end{document}
