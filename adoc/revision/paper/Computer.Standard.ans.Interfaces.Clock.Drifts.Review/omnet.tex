OMNeT++ \cite{omnet2008} is a versatile C++ simulation library and framework specifically designed for building network and Cyber-Physical simulators. It offers extensibility, modularity, and a component-based approach. The modeling framework comprises two essential components: the NED topology description language and the component code written in C/C++. The NED modeling language is composed of a set of components and connections. We reuse the function \emath{eval} defined previously such that \emath{ eval: \vartheta \rightarrow \mathbb{D}}. The NED Component is defined as follows: 

\begin{mydef} \label{def:omnetcomponent} A NED component \quot{\emath{\mathcal{ND}}} is a tuple \emath{\mathcal{ND}= \langle \vartheta, \Sigma, Bh\rangle}, where: 
 \begin{itemize}
\item \emath{\vartheta = \vartheta_{p} \cup \vartheta_{m} \cup \vartheta_{l}}  is a finite set of component variables  
 \begin{itemize}
\item  \emath{\vartheta_{p}} is a finite set of parameters associated with the component,
\item \emath{\vartheta_{m}} is a finite set of messages associated with the component. Also, we consider the \emath{X} as valuation over \emath{\vartheta_{m}} and \emath{X'} as its new valuation.
\item \emath{\vartheta_{l}} is a finite set of local variables associated with the component.
\end{itemize}
\item \emath{\Sigma} is a finite set of gates that can take the form of \emath{input} (formally ?) and \emath{output} (formally !). We identify by \emath{\tau} as internal behavior related to the OMNeT component.
\item \emath{Bh} function is implemented in C/C++ and is responsible for performing specific operations on the received/sent message.
\end{itemize}
\end{mydef}

To gain a deeper understanding of the OMNet++ component behavior, we establish semantics for the  \emath{\mathcal{ND}} state based on component variables as follows: 

\begin{mydef}[\emath{\mathcal{ND}} state] A \emath{\mathcal{ND}} state \quot{\emath{s}} is the valuation \emath{\theta} of  \emath{\mathcal{ND}} components variables such that: \emath{s= \langle x_{0}, \ldots, x_{n}, \theta\rangle} such that \emath{\theta \models g} and \emath{g} is the guard over states values.
\end{mydef}

Each \(\mathcal{ND}\) component is capable of establishing communication with other \(\mathcal{ND}\) components by utilizing its gates to create a composite component. The definition is outlined as follows:
% The formalism of OMNeT++ can be encompassed within the formalism of Probabilistic Automata (PA)  as follows:


% \begin{mydef} A PA of a OMNeT++ model is represented as a tuple \emath{\mathcal{N}_{\mathcal{P}} =\langle  S_{0}, S, \Sigma, AP, L, \delta\rangle}, where:
% \label{ts}
% \begin{itemize}
% 	\item \emath{S_{0} = (\langle x_{0},\ldots,x_{n}, \theta \rangle )} is an initial state that corresponds to the initial variable values.
 
% 	\item \emath{S=\mathbb{D}^{I}} is the domain of \emath{\mathcal{N}} features \emath{I},

% 	\item \emath{\Sigma}  is a finite set of gates,

%     \item \emath{AP = S \cup G},

%  	\item \emath{L(\langle s,\theta\rangle ) = \{ s\} \cup \{ g \in G\} | \theta \models g}, and

%     \item \emath{\delta : S \times \Sigma \rightarrow Dist(S)} is probabilistic transition such that \emath{\lambda \in Dist(S)} is equal to 1. The transition takes the form: \emath{s \gparrow{\alpha}_{\lambda} s'} where \emath{\alpha \in \Sigma}
% \end{itemize}
% \end{mydef}


\begin{mydef}[Interaction] \label{inter} An interaction in OMNeT++ refers to a realized connectivity activity between components. It can be represented as a tuple  \emath{\langle \Sigma_a, d_a\rangle}, where:  \emath{\Sigma_{a} \subseteq \Sigma} represents a set of gates that are associated with the interaction \emath{a}, and \emath{d_{a}} is a delay associated with the interaction.
\end{mydef}




The result of the interaction is a composition of synchronized components obtained by using the component composition operator  $\gamma$  presented in Definition \ref{def:composition}.


\begin{mydef}[Composition]\label{def:composition} The composition \emath{\gamma} (\emath{\mathcal{ND}_{1},\ldots,\mathcal{ND}_{n}}) of \emath{n}  components is a network component resulting from \(\mathcal{ND}\) composition on interaction \emath{\Sigma_{a} \subseteq \Sigma} in Definition \ref{inter} represented by \ref{cmp1}  and \ref{cmp2} operational semantics rules:
\begin{boxD}
   \begin{equation}\frac{ s_{0} \gparrow{\tau} s'_{0}} { \langle s_{0},\ldots,s_{n}\rangle \xrightarrow{\tau}\langle s'_{0},\ldots,s_{n}\rangle  } \label{cmp1} \tag{\emph{Internal Update}} 
   \end{equation}
   
   where  \emath{X'_{\vartheta_{l}}=X[v_{l}:=bh(l)]} 
   \begin{equation}\frac{ s_{0} \gparrow{a!m} {s'_{0}} \wedge s_{1} \gparrow{a?m} {s'_{1}} } { \langle s_{0},\ldots, s_{1}, \ldots,s_{n}\rangle \xrightarrow{a}\langle s'_{0},\ldots, s'_{1},\ldots,s_{n}\rangle  } \label{cmp2} \tag{\emph{Send/receive}} \end{equation}
              where  \emath{X'_{\vartheta_{m}}=X[v_{m}:=m]} 
              \end{boxD}
\end{mydef}