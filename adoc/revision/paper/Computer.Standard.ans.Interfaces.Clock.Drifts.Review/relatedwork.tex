
\cmt{This section lays the groundwork by exploring the fundamentals of clock drift, also known as clock deviation (both terms are interchangeable). We will delve into existing research activities focused on modeling and resolving clock deviation.}


\subsection{Clock drifts}
\label{drifting}
Several research papers have explored the phenomenon of clock drift caused by variations within internal clocks. In this work, we build upon the research established in \cite{WEBSTER2020101183} and \cite{Elsts2018}.  We categorize the different sources of clock variation into three main categories: Temperature Variations, Standard-Specific Drift, and Production Spread.  \cite{Zaid2012}.


\subsubsection{Clock drift caused by temperature variations}
\label{clockdriftliterature}
According to \citeauthor{WEBSTER2020101183} \cite{WEBSTER2020101183} \cite{WebsterBDFM18}, differences in operating temperature can affect drone clock synchronization. If the clocks have different temperatures, the warmer clock will lag behind the colder clock by one tick. This means the colder clock will have ticked twice while the warmer clock has ticked only once, likely due to oscillator variations in \cite{Lenzen2009} as \citeauthor{Lenzen2009}  observed similar phenomena on the Mica2 processor. Hence, on average, after every \texttt{230,468} ticks, the warmer clock will fall behind the colder one by exactly one tick. This was then translated into a probability of 1 in \texttt{230,468}, equivalent to \texttt{0.000004339}.

\subsubsection{Standard-specific clock drift}
According to the IEEE 802.15.4 standard and the work established by \citeauthor{Elsts2018} \cite{Elsts2018} set a revision definition of maximum resynchronization period between wireless devices as it refers to the longest interval between synchronizations a network can tolerate while ensuring reliable communication. Three factors are identified in this standards, the guard time \emath{t_{g}} (This is the time a receiver listens for a packet beyond the expected arrival time to account for clock drift), the preamble transmission time  \emath{t_{p}} (This is the time it takes to transmit the preamble, a signal preceding the actual data in a packet), and clock drift (\emath{\delta}). The following equation relates the maximum resynchronization period (\emath{\Delta t}) to guard time and clock drift: \emath{\Delta t\leq (t_{g} - 2*t_{p} )/2*\delta}. To illustrate these clock variations, a Python code is available in section artifacts that simulate drone resynchronization. This code extracts the average clock drift, which will be used for verification and validation purposes. The relevant Python code can be found in \cite{csi2023} under the model reference \eclipse{P1}. The computed average clock drift is approximately \texttt{0.44} (parts per million). 

\subsubsection{Clock drift caused by production spread}
Clock drift caused by production spread refers to the slight variations in clock frequency that occur between individual devices despite being manufactured identically. The manufacturer typically specifies the range (bounds) of these deviations. According to \cite{Elsts2018}, when using oscillators, a tolerable drift in the worst-case scenario can be as high as \texttt{0.40} parts per million (ppm).



\subsection{Reasoning on clock drifts}

This literature review comprehensively surveys research on clock deviation or clock drift phenomena in networking. As discussed in Section~\ref{clockdriftliterature}, \citeauthor{Elsts2018} \cite{Elsts2018} provide a comprehensive foundation for understanding clock drift related to temperature variations, standard-specific deviations, and production spread. These phenomena are all validated through oscillator observations. We rely on those parameters to build our models in OMNeT++ and PRISM. Moreover, the survey provided by \citeauthor{Amulya20215} \cite{Amulya20215} \cmt{to better understand clock synchronization protocols in wireless sensor networks (WSNs) from the perspective of how and when clock values are propagated within the network and how physical clocks are updated}. \citeauthor{HAUWEELE20201} \cite{HAUWEELE20201} study investigates clock drift in Wireless Sensor Networks (WSNs) and its impact on Radio Duty Cycle (RDC) protocols \cite{Bezunartea2017}. The authors address limitations in the COOJA simulator \cite{CoojaWebsite}, which cannot accurately reproduce the clock drift. They propose a new mathematical model that allows them to simulate clock drift that causes periodic communication blackouts – a phenomenon previously impossible to replicate within COOJA. Several research works focus on achieving fast synchronization between physical clocks in Wireless Sensor Networks (WSNs) as in \cite{XIE2018133};  \citeauthor{XIE2018133} ensure a valid clock while it requires minimal computation communication from each node as it is suitable for large-scale WSNs due to its fast convergence.  \citeauthor{Xuxin2019}  \cite{Xuxin2019} propose a robust maximum time synchronization (RMTS) scheme for clock synchronization in the presence of both noise and packet loss. This scheme leverages the concept of maximum consensus and is theoretically proven to be effective. The authors also validate its performance through simulation. \cmt{\citeauthor{li2016} \cite{li2016} focuses on the challenges of verifying timed security protocols when clock drift is present. It proposes an extension of the timed applied pi-calculus formalism to model clock drift and presents an automated verification approach for security properties \cite{Bagnara2022}. \citeauthor{Zaid2012}\cite{Zaid2012} \cmt{proposes a probabilistic model checking approach to verify Clock Domain Crossing (CDC) interfaces. It models CDC interfaces as Markov Decision Processes (MDP) and uses PCTL properties to verify their correctness. The MDP incorporates a synchronizer with various states to mitigate metastability-related failures.} \citeauthor{Varalakshmi2013}\cite{Varalakshmi2013} introduce a novel clock synchronization technique for clients and servers in the cloud. This synchronization strengthens a \quot{port hopping} method designed to prevent Denial-of-Service (DoS) attacks on open server ports. By knowing the open ports for a specific timeframe, it effectively thwarts DoS attacks on those ports. \citeauthor{Tsurumi2022} \cite{Tsurumi2022} propose a clock drift estimation and compensation scheme for PLIM that eliminates any additional computational burden on end nodes (ENs). This is achieved by performing the entire operation at the gateway (GW) side. The authors validate their approach through both experimental testing and real-world implementation on a commercial LoRaWAN system. \citeauthor{Lübken2024} \cite{Lübken2024} address time synchronization in space WSNs. Harsh environments cause clock drift between sensor nodes, hindering low-latency data acquisition. The paper proposes a novel Adaptive Local Clock Control (ALCC) algorithm that significantly outperforms existing methods by minimizing clock deviation under these extreme conditions. The effectiveness of ALCC is validated through simulations and hardware testing. \citeauthor{Kaburaki2024} \cite{Kaburaki2024} tackle data collection challenges in sensor networks using Low-Power Wide-Area Networks (LPWANs). The issue arises from collisions caused by both simple access schemes and clock drift in these networks. The authors propose a resource allocation scheme to avoid packet collisions under the effect of clock drift. Simulations show this method significantly improves packet delivery rates compared to existing approaches.}


\cmt{Traditionally, clock synchronization research relies heavily on simulation \cite{Amulya20215, HAUWEELE20201, XIE2018133, Xuxin2019,  Varalakshmi2013, Tsurumi2022, Lübken2024, Kaburaki2024, Akpinar2020} for performance evaluation, model checking for the feasibility of operations under clock drifts \cite{WEBSTER2020101183,Zaid2012,li2016}, and estimation methods \cite{Arao2017}. However, there is a lack of studies investigating the applicability of simulation and model checking (Validation \& Verification) tools to assess the efficiency of synchronization algorithms compared to existing tools. While works like \cite{WEBSTER2020101183} provide a strong foundation for mathematically building synchronization protocols, their reliance on abstraction to manage state space explosion can be a disadvantage. Our approach explores synchronization through simulation and model checking but then learns models that accelerate reasoning. This is achieved not through abstraction, but by learning from streaming simulation data. The resulting models can then be verified using properties expressed in temporal logic.} 