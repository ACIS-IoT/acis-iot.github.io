In this section, we explore the formalisms that capture the underlying semantics associated with clock drifts. Additionally, we will provide the formalization of Probabilistic Decision Trees (PDTs).


\subsection{Modeling formalism}
The modeling formalism presented in this paper primarily relies on PRISM. However, we will emphasize the mathematical foundations to acquaint readers with \cmt{Probabilistic Automata (PA)}.

\subsubsection{The PRISM language}
\label{sec:prism:rappel}
We consider \cmt{PA} in PRISM language \cite{Kwiatkowskaprism2011} as a modeling formalism to capture the semantics of stochastic automata and express the non-determinism in wireless sensor networks. 

%\subsection{Introduction to PRISM}
The PRISM model \emath{\mathcal{P}} is represented by a composition of modules on actions. Modules are characterized by variables and commands where the state of the module is the variable valuations. The behavior of each module is described by a set of commands (i.e., transitions). A command takes the form: \emath{ [a] g \rightarrow \lambda_{1}: u_{1} + \ldots+ \lambda_{n}: u_{n} } or, \emath{[a] g \rightarrow u}, which means, for the action \quot{\emath{a}} if the guard \quot{\emath{g}} is true, then, an update \quot{\emath{u_{i}}} is enabled with a probability \quot{\emath{\lambda_{i}}}. A guard is a condition that determines whether a command can be executed based on evaluations of variables using logic operators. The update \quot{\emath{u_{i}}} is an evaluation of variables \emath{v_{i}}; evaluated via expressions denoted by \quot{\emath{\theta}} such that \emath{ \theta: V \rightarrow \mathbb{D}}, where \emath{\mathbb{D}} is a domain of variables such that \emath{\mathbb{D}=\mathbb{N} \cup \{true, false\} }.


\begin{mydef}[\emath{\mathcal{P}} state] \label{def:prismstate} A state of \emath{\mathcal{P}}  is the valuation  of  \emath{\mathcal{P}} modules variables \emath{ v_{0}, \ldots, v_{n}} in the form \emath{s= \langle x_{0}, \ldots, x_{n}, \theta\rangle} such that \emath{\theta \models g} and \emath{g} is the guard over states values., where \emath{x_{i}} is the valuation of variable \emath{v_{i}}.
\end{mydef}



\subsubsection{Probabilistic automata}
\label{sec:pa}


PA \cite{ref27} is a modeling formalism that exhibits probabilistic and nondeterministic features \cmt{implemented by PRISM}. Definition \ref{def:pa} formally illustrates a PA where \emath{Dist(S)} denotes the set of convex distributions over the set of states $S$ and \emath{\mu} is a distribution in \emath{Dist(S)} that assigns a probability \emath{\mu(s_{i})=p_{i}} to the state \emath{s_{i} \in S}.

\begin{mydef} \label{def:pa} \normalfont  A \cmt{PA} is a tuple \emath{\mathcal{M} =\langle  s_{0}, S, \Sigma, AP, L, \delta\rangle}:
\label{ts}
\begin{itemize}
	\item \emath{s_{0}} is an initial state, such that \emath{s_{0} \in S},
 
	\item \emath{S} is a set of states,

	\item \emath{\Sigma}  is a finite set of actions, communicating between automata is based on notation (\emath{!}) for sending and (\emath{?}) for receiving communicating data.

 	\item \emath{L: S \rightarrow 2^{AP}} is a labeling function that assigns each state \emath{s \in S}  to a set of atomic propositions taken from the set of atomic propositions (\emath{AP}), and

    \item \emath{\delta : S \times \Sigma \rightarrow Dist(S)} is a probabilistic transition function assigning for each \emath{s \in S} and \emath{\alpha \in \Sigma } a probabilistic distribution \emath{\mu \in Dist(S)}. The operational semantics rules in \fig{fig:prism:rules} model all generic probabilistic transitions.
\end{itemize}
\end{mydef}


For PA's composition, this concept is modeled by the parallel composition \cite{ref27}. During synchronization, each PA resolves its probabilistic choice independently \cite{ref27}. For transitions, \emath{s_{1} \xrightarrow[]{ \alpha} \mu_{1}}  and \emath{s_{2} \xrightarrow[]{ \alpha} \mu_{2}} that synchronize in \emath{\alpha} then the composed state (\emath{s'_{1},s'_{2}}) is reached from the state (\emath{s_{1},s_{2}}) with probability (\emath{\mu_{1}(s'_{1}) \times \mu_{2}(s'_{2})}).  In the no synchronization case, a PA takes a transition where the other remains in its current state with probability one.


Performing probabilistic model checking relies on \cmt{PA} and properties expressed in Probabilistic Computation Tree Logic (PCTL)~\cite{hanssonlogic1994}, \cite{ref27}, \cite{goosit1995}. PCTL enables the expression of probabilistic queries such as \quot{\emph{What is the probability that the system will eventually reboot ?}}, expressed as P =? [F reboot]. In this context, \quot{reboot} is the label that refers to the states indicating a system reboot.


\noindent
\begin{figure*}[!ht]
    \centering
	\begin{center}
	
		\begin{tabular}{ |m{15cm}| }			\hline
    			\\ [1.5ex]
    		\begin{itemize}
    		
	        \setlength\itemsep{1.5em}

        \item \textbf{Update}. This axiom describes the probabilistic local updates for variable \emath{v_{i}}: $$\frac{ s_{i} \xrightarrow{g:a}_{\lambda_{i}} {s'_{i}} \wedge \theta \models g } { \langle s_{0},\ldots,s_{i}, \ldots,s_{n},\theta\rangle \xrightarrow{a}_{\lambda_{i}}\langle s_{0},\ldots,s'_{i}, \ldots,s_{n},\theta'\rangle  }  $$
              where  \emath{\theta'=\theta[v_{i}:=effect(v_{i})]}

         \item \textbf{Synchronization}. This axiom permits the synchronization between modules on a given action a: $$\frac{ s_{i} \xrightarrow{g_{1}:a}_{\lambda_{i}}s'_{i} \wedge \theta \models g_{1} \wedge s_{j} \xrightarrow{g_{2}:a}_{\lambda_{j}}s'_{j} \wedge \theta \models g_{2} } { \langle s_{0},\ldots,s_{i},\ldots,s_{j}, \ldots,s_{n},\theta\rangle \xrightarrow{a}_{\lambda_{i} \times\lambda_{j}}\langle s_{0},\ldots,s'_{i},\ldots,s'_{j}, \ldots,s_{n},\theta'\rangle }  $$   where  \emath{\theta'=\theta[v_{i}:=effect(v_{i})]}      

        \item \textbf{Send/receive}. This axiom describes synchronous data exchange:  $$\frac{ s_{i} \xrightarrow{g_{1}:a!v_{i}}_{\lambda_{i}}s'_{i} \wedge \theta \models g_{1} \wedge s_{j} \xrightarrow{g_{2}:a?v_{j}}_{\lambda_{j}}s'_{j} \wedge \theta \models g_{2} } { \langle s_{0},\ldots,s_{i},\ldots,s_{j}, \ldots,s_{n},\theta\rangle \xrightarrow{a}_{\lambda_{i} \times\lambda_{j}}\langle s_{0},\ldots,s'_{i},\ldots,s'_{j}, \ldots,s_{n},\theta'\rangle }  $$   where  \emath{\theta'=\theta[v_{j}:=v_{i}]}  
              
        \end{itemize}
      \\
		\hline
		\end{tabular}
	\end{center}
    \caption{PRISM-PA Semantics Rules.}
    \label{fig:prism:rules}
\end{figure*} 
\normalsize





% and effect such that\emath{ effect: V \rightarrow \mathbb{D}} (Utilized for evaluating module variables).
% Let \emath{\mathbb{D}} be a domain of variables such as \emath{\mathbb{D}=\mathbb{N} \cup \{true,false\} }. We define valuations for variables as \quot{\emath{X}} such that \emath{ X: V \rightarrow \mathbb{D} } that associate each variable in \emath{V} with a value in \emath{\mathbb{D}}. We use \emath{x_{0}, x_{1},\ldots, x_{n}} as new valuations of variables \emath{v_{0}, v_{1},\ldots, v_{n}}. In the PRISM module, each local variable is initialized, and then we define a function \emath{init: V \rightarrow \mathbb{D}} that assigns an initial value in \emath{\mathbb{D}} for variables in \emath{V}. 

% Each PRISM command is encapsulated within a PRISM module defined as follows:

% \begin{mydef} \label{def:prismmodule} \normalfont A PRISM module \quot{\emath{\mathcal{D}}} is a tuple \emath{\mathcal{D}= \langle \vartheta_{L}, C\rangle}, where: 
%  \begin{itemize}
% \item \emath{\vartheta_{L}} is a finite set of local variables associated with the module \emath{\mathcal{D}} initialized with \emath{init},
% \item \emath{C} is a finite set of commands that defines the behavior of module \emath{\mathcal{D}}. Formally, we consider the command of the form \emath{ [a] g \rightarrow \lambda: u} as \emath{x_{i}\xrightarrow{g:a}_{\lambda}x'_{i}} such that \emath{x_{i}\models g}, \emath{g \in Const(V)} and \emath{X':=X[v_{i}:=eval(v_{i})]}. 
%             \end{itemize}
% \end{mydef}


% \subsubsection{PRISM-PA semantics}
% Definition \ref{def:prismpa}  stipulates the formal definition of PRISM probabilistic automata denoted by  \emath{\mathcal{M}_\mathcal{P}}. The stepwise behavior of \emath{\mathcal{M}_\mathcal{P}} follows the operational semantic rules provided in \fig{fig:prism:rules}. However, we would like to highlight the concept of channeling as described by \cite{baierprinciples2008} , where communication is depicted using the Communicating Sequential Processes (CSP) notation. In this notation, the symbol "!" represents the sending operation, while the symbol "?" represents the receiving operation.


% \begin{mydef} \label{def:prismpa}\normalfont (PRISM-PA) A probabilistic automata of a PRISM model \emath{\mathcal{P}} is a tuple  \emath{M\mathcal{_{P}}=\langle  s_{0}, S,  AP, L, \Sigma, \delta \rangle}:
% \begin{itemize}

% 	\item \emath{s_{0} = (\langle x_{0},\ldots,x_{n}, \theta \rangle )} is an initial state such that \emath{ x_{0}\wedge\ldots\wedge x_{n}\wedge \theta \models g_{0}}, \emath{g_{0}} is the initial guard to be satisfied by the initial state.
 
% 	\item \emath{S= \mathbb{D}^{V}} is a finite set of states reachable from \emath{s_{0}},

%  	\item \emath{AP= V \cup Const(V)} is a set of variable valuations,
  
% 	\item \emath{L(\langle x_{0},\ldots,x_{n} \rangle) = \{ x_{0},\ldots,x_{n} \} } such that \emath{[[g(X)]]=\top} is a labeling function,
 
% 	\item \emath{\Sigma}  is a finite set of actions, and  

%     \item \emath{\delta} is defined by the PRISM commands C modeled by the operational semantics rules in \fig{fig:prism:rules}.
% \end{itemize}
% \end{mydef}


\subsection{Probabilistic Decision Trees}
\includesection{pdt}