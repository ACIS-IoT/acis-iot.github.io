The literature review mentions a limited number of papers related to the formal verification of the RabbitMQ \cmt{broker}. However, it highlights the presence of research addressing validation through simulation. The RabbitMQ \cmt{broker} offers a range of robust features, including Authentication and Access Control \cite{rabbitmq-access-control}. This is achieved using JWT-encoded OAuth 2.0 access tokens \cite{rfc6749} and certificate-based client authentication. Another critical feature is Clustering, which ensures high availability and fault tolerance \cite{rabbitmq-clustering}. In the event of a node failure, clients can reconnect to an alternate node, recover their topology, and resume their operations uninterrupted. A wealth of related features can be found on the RabbitMQ documentation website \cite{rabbitmq-docs}. The RabbitMQ \cmt{broker} has undergone validation and verification in multiple studies, highlighting its distinguished features \cite{Li2020, Li2022, Ionescu2015, Hong2018, Bagaskara2020, Rostanski2014}.

The paper by  \citeauthor{Li2020} \cite{Li2020} presents a formal verification of the RabbitMQ \cmt{broker} using timed automata and the UPPAAL model checker\cite{behrmann2006uppaal}. Essential properties, including Reachability of Data and Message Acknowledgement, are successfully verified, confirming RabbitMQ's adherence to these properties. An extension of this work is presented in a subsequent article by \citeauthor{Li2022}  \cite{Li2022}, which emphasizes the integration of the Kerberos network authentication protocol \cite{Neuman1994}  to ensure secure communication (related to authentication). Using UPPAAL \cite{behrmann2006uppaal}, the authors model and verify the enhanced protocol, providing evidence of RabbitMQ's ability to maintain secure communication.

In the paper by \citeauthor{Ionescu2015} \cite{Ionescu2015}, a performance comparison of RabbitMQ and ActiveMQ \cite{activemq} brokers in message-oriented middleware applications is conducted, with a specific focus on message sending and receiving. The analysis reveals distinctions between the two brokers: ActiveMQ exhibits superior message reception speed, while RabbitMQ demonstrates qualitative message delivery to clients due to its implemented security functions during reception. Furthermore, \citeauthor{Rostanski2014} \cite{Rostanski2014}  investigate design considerations for scalability and high availability in RabbitMQ. The study explores the usage of clustered RabbitMQ nodes and mirrored queues and presents simulation test results to evaluate performance.

During our review of the relevant literature, we came across the work conducted by \citeauthor{Li2020} \cite{Li2020}, which primarily focuses on the functional properties of RabbitMQ using UPPAAL. An extension of their work is presented in \citeauthor{Li2022}  \cite{Li2022}, where they specifically address verification of unauthorized access to the communication channel. In contrast, studies such as \cite{Ionescu2015, Hong2018, Bagaskara2020, Rostanski2014} primarily utilize simulation techniques to assess performance aspects of deployed servers, including scalability, memory usage, and throughput. However, these studies do not consider the security aspects related to unauthorized data modification given the complexity of the architecture. In contrast, our paper aims to bridge this gap by providing a comprehensive approach to modeling and analyzing unauthorized data modification attacks within the considered system architecture. We accomplish this by employing a game model that captures concurrent access through Concurrent Stochastic Games (CSGs) using the PRISM games.


\cmt{We rely on a straightforward algorithm that utilizes Python libraries when extracting attack frequency and optimizing techniques. This algorithm allows us to calculate the Mean Time Between Attacks (MTBA) from the dataset using established algorithms from the literature, such as the Broyden–Fletcher–Goldfarb–Shanno algorithm (L-BFGS-B) \cite{liu1989limited} and the Nelder-Mead algorithm \cite{nelder1965simplex}. These algorithms have been widely used in the field and are known for their effectiveness in this context. Other nature-inspired metaheuristic algorithm, specifically focusing on optimization techniques that could be used in such optimization.  In \citeauthor{Genghis2023} in \cite{Genghis2023} introduces a nature-inspired metaheuristic algorithm called GKS optimizer (GKSO), inspired by the behavior of the Genghis Khan shark. GKSO  performs optimization tasks achieved by simulating hunting, movement, foraging, and self-protection mechanisms. The algorithm's viability and superiority are validated through qualitative and quantitative analyses, demonstrating its exploration and exploitation capabilities. \citeauthor{Ezugwu2022} in  \cite{Ezugwu2022} introduces prairie dog optimization (PDO), a nature-inspired metaheuristic algorithm that mimics the behavior of prairie dogs in their natural habitat. PDO utilizes foraging and burrow-build activities for exploration while exploiting the prairie dogs' communication skills to converge toward promising locations. \citeauthor{Jeffrey2022} in  \cite{Jeffrey2022} presents the dwarf mongoose optimization (DMO) algorithm inspired by the foraging behavior of dwarf mongooses. DMO utilizes three social groups (alpha group, babysitters, and scout group) to mimic the mongoose's foraging strategy. \citeauthor{Agushaka2023} in  \cite{Agushaka2023} introduces the Gazelle Optimization Algorithm (GOA), inspired by the survival behavior of gazelles in predator-dominated environments. The algorithm incorporates two phases: exploitation, where gazelles graze peacefully, and exploration, evading predators.} 

\cmt{\citeauthor{DETDO2023} in \cite{DETDO2023} presents DETDO, an adaptive hybrid dandelion optimizer based on the Dandelion Optimizer (DO). DETDO combines adaptive tent chaotic mapping, differential evolution strategy, and adaptive t-distribution perturbation to prevent local optima and improve convergence speed. \citeauthor{Mojtaba2024} in \cite{Mojtaba2024} introduces Lung performance-based optimization (LPO) inspired by the performance of lungs in the human body. LPO draws inspiration from the adaptability and optimization of the respiratory system in solving complex optimization problems.}