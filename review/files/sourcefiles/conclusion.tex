%This paper has presented a comprehensive modeling of the RabbitMQ architecture in CSG, abstracting it into four entities: southbound, exchange, queues, and fog, while considering both attackers and sensors. Based on this model, we have successfully verified security properties concerning threats related to payloads and message routing keys. Furthermore, we have highlighted the importance of attack mitigation mechanisms supported by the protocol. In future research, we intend to investigate the state space exploration resulting from the model construction in PRISM-games, further enhancing our understanding of the system's security aspects.

In this paper, we presented a formal specification and analysis of the RabbitMQ architecture from a security perspective in a Concurrent Stochastic Game, leveraging reward Probabilistic Alternating Temporal Logic (rPATL) to model security requirements as game goals. Based on this model, \cmt{we have verified security properties concerning threats related to payloads and message routing keys}. An evaluation of the proposed formalization is presented through a practical application to a use case in the IoT domain facing DoS, MITM ARP spoofing, and Mirai attacks at southbound bridges (sensors) and queues using PRISM-games model checker. Furthermore, we have highlighted the importance of attack mitigation mechanisms supported by the protocol. 

\cmt{We identified some limitations to consider. Firstly, the PRISM currently has limited language constructs. It only supports integer values within modules, which may restrict the expressiveness of our models. Secondly, the paper does not fully address the model's scalability with increasing module numbers. While including features like enqueuing and dequeuing, further investigation is needed to ensure the model remains efficient and performant as the system grows in complexity.}


\cmt{In our forthcoming research endeavors, we intend to explore the state space resulting from constructing the model in PRISM-games. This exploration will deepen our understanding of the system's security at different levels of refinements and abstractions. Further, we aim to investigate the integration of our formalization process with established Model-Based Development (MBD) methodologies and tools.  Also, we plan to implement various optimization techniques, as identified in the related work, to calculate attack frequencies. Moreover, we intend to model the potential impacts of these attacks on the availability and reliability of the protocol.}